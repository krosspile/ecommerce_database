\section{Descrizione}

Si vuole progettare una base di dati per la\textbf{ gestione dell'attività lavorativa di un e-commerce di abbigliamento}. Lo shop online deve essere accessibile a un qualsiasi visitatore, che può effettuare ordini solo una volta registrato. I dati richiesti durante la registrazione sono \textit{nome}, \textit{cognome}, \textit{email} e \textit{password}. \textbf{Il cliente ha la possibilità di creare un ordine} contenente uno o più \textit{prodotti}, specificando un \textit{indirizzo di spedizione}. \textbf{Ogni prodotto appartiene a una determinata categoria} (magliette, pantaloni o scarpe) ed è caratterizzato da attributi specifici tipici di un capo d'abbigliamento (\textit{modello}, \textit{marca}, \textit{prezzo}, \textit{taglia}, \textit{colore}), inoltre questi si distinguono in\textit{ capi maschili}, \textit{femminili} o \textit{unisex}. \textbf{In alcuni periodi potrebbe essere applicato uno sconto ad alcuni articoli}, in questo caso si deve tenere conto del \textit{prezzo base}, \textit{prezzo scontato} e della\textit{ data di inizio} e \textit{fine promozione}.
\textbf{L'utente registrato deve avere la possibilità di aggiungere una recensione} ad un articolo, nel caso in cui quest'ultimo sia presente in un qualsiasi precedente acquisto. Il feedback prevede una valutazione da 1 a 5 stelle (\textit{rating}) e un \textit{commento}. Ogni prodotto deve avere una \textit{valutazione complessiva}, che equivale alla media di tutte le recensioni lasciate fino al momento in cui questa viene consultata.
\textbf{Lo shop online deve fornire assistenza al cliente in caso di problemi}: l\textbf{'utente deve avere la possibilità di aprire una segnalazione} riguardante un acquisto. \textbf{Il personale dell'e-commerce si occupa di fornire assistenza ai clienti} e il database tiene conto del loro \textit{nome}, \textit{cognome}, \textit{data di assunzione},\textit{ codice fiscale} e \textit{segnalazioni gestite}; \textit{queste ultime indicano approssimativamente il grado di esperienza dell'operatore}. Una segnalazione deve permettere al Cliente di fornire una \textit{descrizione} specifica del problema e si ritiene fondamentale memorizzare i \textit{timestamp} di apertura e chiusura segnalazione. Il dipendente che si occupa della segnalazione, in base alla gravità,\textbf{ può decidere se fornire un rimborso al cliente}, ma in qualsiasi caso deve scrivere un breve \textit{report} che descrive com'è stato risolto il problema.
\newpage