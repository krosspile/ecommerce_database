\section{Descrizione}

Si vuole progettare una base di dati per la\textbf{ gestione dell'attività lavorativa di un e-commerce di abbigliamento}. Lo shop online deve essere accessibile a un qualsiasi visitatore, che può effettuare ordini solo una volta registrato. I dati richiesti durante la registrazione sono nome, cognome, email e password. \textbf{Il cliente ha la possibilità di creare un ordine} contenente uno o più prodotti, specificando un indirizzo di spedizione. Ogni prodotto appartiene a una determinata categoria (magliette, pantaloni o scarpe) ed è caratterizzato da attributi specifici tipici di un capo d'abbigliamento (modello, marca, prezzo, taglia, colore), inoltre questi si distinguono in capi maschili, femminili o unisex. In alcuni periodi potrebbe essere applicato uno sconto ad alcuni articoli, in questo caso si deve tenere conto del prezzo base, prezzo scontato e della data di inizio e fine promozione.
\textbf{L'utente registrato deve avere la possibilità di aggiungere una recensione} ad un articolo, nel caso in cui quest'ultimo sia presente in un qualsiasi precedente acquisto. Il feedback prevede una valutazione da 1 a 5 stelle e un commento opzionale. \textbf{Ogni prodotto deve avere una valutazione complessiva}, che equivale alla media di tutte le recensioni lasciate fino al momento in cui questa viene consultata.
\textbf{Lo shop online deve fornire assistenza al cliente in caso di problemi}: l\textbf{'utente deve avere la possibilità di aprire una segnalazione} riguardante un acquisto. Il personale dell'e-commerce si occupa di fornire assitenza ai clienti e il database tiene conto del loro nome, cognome, data di assunzione e segnalazioni gestite, le quali indicano approssimativamente il grado di esperienza dell'operatore. Una segnalazione deve permettere al Cliente di fornire una descrizione specifica del problema e si ritiene fondamentale memorizzare i timestamp di apertura e chiusura segnalazione. Il dipendente che si occupa della segnalazione, in base alla gravità, può decidere se fornire un rimborso al cliente, ma in qualsiasi caso deve scrivere un breve report che descrive com'è stato risolto il problema.
\newpage